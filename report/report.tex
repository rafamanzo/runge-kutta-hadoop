\documentclass[a4paper,11pt]{report}
\usepackage[T1]{fontenc}
\usepackage[utf8]{inputenc}
\usepackage{lmodern}
\usepackage[brazil]{babel}
\usepackage{hyperref}

\title{Relatório do segundo exercício programa de MAC0431}
\author{Giancarlo Rigo\\
        Rafael Reggiani Manzo}

\begin{document}

\maketitle
\tableofcontents

\begin{abstract}
Para este exercício programa pretendemos implementar o algoritmo de integração numérica Runge-Kutta\footnote{PRESS, William H. et al. \textit{Numerical Recipes in C}}. Este é um algoritmo que dados uma equação diferencial e um ponto inicial, aproxima o valor da função objeto da equação diferencial. Quando são dados mais de um ponto inicial o método é paralelizável para cada um destes pontos que possuem resultados idenpendentes para o método.

Este é o mesmo tema de nosso trabalho de conclusão de curso, onde já temos este método implementado em \textit{C++} com \textit{POSIX Threads}, \textit{CUDA} e \textit{OpenCL}\footnote{\href{https://github.com/rafamanzo/runge-kutta}{https://github.com/rafamanzo/runge-kutta}}.

Uma das aplicações deste método, é a reconstrução de fibras do corpo humano a partir de imagens por difusão de tensores\footnote{DTI - Diffusion Tensor Images} por ressonância magnética. Neste contexto, estamos implementando o método em \textit{GPU} para possibilitar a reconstrução em tempo real e compará-la com a em \textit{CPU}.

Assim, como este problema não possui dependência dos dados entre cada instância e envolve grandes quantidades de dados, parece ser bastante plausível para também ser implementado com o \textit{Apache Hadoop}. Além disto será mais uma informação complementar para utilizarmos como comparação em nossa monografia.
\end{abstract}

\chapter{Compilação e execução}

\section{Ambiente}
O software foi desenvolvido em Ubuntu 12.04 64bits para a versão 1.0.4 do hadoop.

\section{Compilação}
O \textit{jar} gerado pelo projeto, uma vez que este excede o limite de espaço (2MB) imposto pelo PACA, encontra-se disponível para download em:\\ \href{https://github.com/downloads/rafamanzo/runge-kutta-hadoop/tracer.jar}{https://github.com/downloads/rafamanzo/runge-kutta-hadoop/tracer.jar}.

Também é possível gerar este arquivo através do script Ant encontrado na raiz do repositório deste projeto também (\href{https://github.com/rafamanzo/runge-kutta-hadoop}{https://github.com/rafamanzo/runge-kutta-hadoop}).

Por fim, outra  opção é importar o projeto para a IDE Eclipse e lá exportá-lo.

\section{Execução}
Para executar o programa, são esperados quatro argumentos: arquivo com o campo vetorial de entrada; tamanho do passo; arquivo com a lista de pontos iniciais (onde o map será feito); arquivo de saída.

Devendo ser da forma:

\begin{verbatim}
  hadoop -jar tracer.jar <field file> <step size>
                         <initial points file> <trajectories file>
\end{verbatim}

\chapter{Implementação}
A implementação foi dividida em dois pacotes. Um, vector, contendo abstrações para facilitar o trabalho com vetores em três dimensões. E outra, core, com as classes específicas para a execução no Hadoop.

Além disto, dentro da pasta lib, foram disponibilizadas todas as bibliotecas necessárias para execução.

\section{Map}
Esta classe ficou extensa devido a implementação das funções auxiliares para o calculo das trajetórias de cada ponto inicial. A função map, para cada ponto inicial contido no arquivo de entrada executa o método obtendo uma trajetória e o manda para a saída como valor e o ponto inicial como chave.

Outra particularidade está na passagem de argumentos. Como o campo vetorial e o tamanho do passo não devem ser operados pelo map, mas são necessários, estes foram passados como parametros da configuração do Job e são recuperados no Map através do método configure que foi sobrescrito.

\section{Reduce}
Classe para simplesmente transformar as trajetorias novamente em texto que será impresso na saída.

\chapter{Analise Hadoop}
É um arcabouço bastante potente, com diversas possibilidades que simplifica a criação de programas distribuídos. Porém com diversos pontos fracos principalmente na organização:

\begin{itemize}
  \item Dependente do Java desenvolvido pela Oracle;
  \item Três versões diferentes disponíveis ao mesmo tempo;
  \item Boa parte dos tutoriais oficiais não especifíca a versão a que se refere;
  \item Documentação sobre a configuração do ambiente de desenvolvimento não cobre todo o processo.
\end{itemize}

Enfim, é uma plataforma muito poderosa para desenvolvimento de aplicações distribuídas, mas que peca na sua documentação e organização. O desenvolvedor acaba tendo de gastar mais tempo configurando o ambiente do que desenvolvendo o software.
\end{document}
